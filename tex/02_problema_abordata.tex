\section {Formularea problemei}

In urma studiului de piata din capitolul anterior am concluzionat ca exista un segment de utilizatori care ar fi interesati in a folosi un astfel de sistem. In cele ce urmeaza voi prezenta 

\section {Studiu asupra realizărilor similare din domeniu}

\section {Stabilirea cerințelor funcționale si nefuncționale ale sistemului}

\subsection{Controlul accesului intr-o cladire}

Scopul principal al acestui sistem este de a oferi sau nu acces intr-o incinta, prin urmare consider aceasta cea mai importanta cerinta functionala.

\subsection{Expunerea unui serviciu REST pentru interfatarea cu alte sisteme}

Expunerea si abstractizarea terminalului \acrshort{pots} este realizata printr-un set de servicii \acrfull{rest} care controleaza starea sa. Acest lucru ne permite interfatarea cu aplicatia mobila, interfata de administrare web si alte servicii precum Google Home/Google Assistant/Apple Home.

\subsection{Implementarea unei functii pentru raspuns automat}

Aceasta functie va permite utilizatorului sa stabileasca o perioada de timp pentru care sistemul va oferi accesul neconditionat.

\subsection{Dezvoltarea unui client mobil Android}

Principalul client care va interactiona cu serviciile \acrshort{rest} va fi aplicatia mobila ce va avea rolul de a notifica userul cand ii suna interfonul si de a controla starea sistemului.

\subsection{Control granular asupra datelor stocate}

Arhitectura aplicatiei necesita interactiunea cu o baza de date, care poate fi tinuta in cloud, pentru convenabilitate sau local.
Folosind tehnologii de containerizare precum Docker, putem stoca baza de date local, informatiile fiind stocate intr-un mediu controlat.

\subsection{Criptarea comunicatiilor cu serviciile web}

Avand in vedere nivelul de acces pe care l-ar oferi un exploit al acestei solutii, comunicatiile intre server si clienti trebuie realizate printr-un canal criptat de tip \acrfull{ssl}. Credentialele userului si ulterior tokenul de acces trebuie trimise doar dupa verificarea autenticitatii serverului si a pachetelor trimise.

\subsection{Oferirea si revocarea accesului la sistem}

Dorim de exemplu sa oferim acces neconditionat unui prieten apropiat pentru a intra in bloc fara a mai suna la interfon. De asemenea ar trebui sa putem realiza si inversul acestei operatii.

\subsection{Expunerea unui flux duplex audio prin tehnologia VoIP}

Pasul final in dezvoltarea acestui sistem ar fi interfatarea cu un \acrfull{adc} si un \acrfull{dac} si expunerea streamurilor de date prin \acrfull{voip}
