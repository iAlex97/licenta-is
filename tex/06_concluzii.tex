\section{Concluzii}

Avand in vedere informatiile expuse in capitolele 1, 2, 3 si cunostintele dobandite in realizarea arhitecturii unui sistem de tip incuietoare inteligenta, dar si rezultatele studiului de caz demonstreaza ca se pot crea solutii folosind tehnologii open-source. Prin alegeri sensibile din punct de vedere tehnic in ceea ce priveste securitatea sistemului, utilizatorul normal poate folosi aplicatia pentru a isi controla interfonul intr-un mod sigur. De asemenea, includerea tehnologiilor open-source inseamna ca utilizatorii avansati au multiple alegeri in ceea ce priveste implementarea solutiei (pot alege unde sunt stocate datele si cine are acces la ele, pot configura din firewall accesul la \acrshort{api} si la interfata administrativa).

Prin urmare, aceasta lucrare arata ca se poate concepe o solutie \acrshort{iot} la nivelul standardelor secolului 21, atat din punct de vedere al securitatii cat si al proprietatii datelor stocate. Integrarea sa usoara poate doar sa beneficieze adoptarii mai rapide a acestui sistem.

\section{Contributii}

\begin{itemize}
  \item Sinteza informatiilor in ceea ce priveste nisa incuietorilor inteligente.
  \item Analiza solutiilor existente din domeniu pentru intelegerea pietii.
  \item Conceperea arhitecturii unui sistem similar care sa satisfaca cerintele functionale.
  \item S-a demonstrat ca se pot folosi componente electronice simple pentru a realiza un optocuplor mai ieftin, dar care ofera acceasi functionalitate in cazul acestei aplicatii.
  \item Implementarea fizica a acestui concept si testarea in viata de zi cu zi.
\end{itemize}

\section{Dezvoltari ulterioare}

Ca dezvoltari ulterioare ale sistemului ar presupune construirea intreaga a unui terminal \acrshort{pots} ca \acrshort{hat} pentru RaspberryPi. Astfel vom putea atinge si cerinta de a oferi utilizatorului fluxuri audio pentru a comunica cu persoana de la celalalt capat al interfonului. 
