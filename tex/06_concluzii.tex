\section{Concluzii 'si contribu'tii}

Având în vedere informa'tiile expuse în capitolele 1, 2, 3 'si cuno'stin'tele dobândite în realizarea arhitecturii unui sistem de tip încuietoare inteligentă, dar 'si rezultatele studiului de caz demonstrează că se pot crea solu'tii folosind tehnologii open-source. Prin alegeri sensibile din punct de vedere tehnic în ceea ce prive'ste securitatea sistemului, utilizatorul normal poate folosi aplica'tia pentru a î'si controla interfonul într-un mod sigur. De asemenea, includerea tehnologiilor open-source înseamnă că utilizatorii avansa'ti au multiple alegeri în ceea ce prive'ste implementarea solu'tiei (pot alege unde sunt stocate datele 'si cine are acces la ele, pot configura din firewall accesul la \acrshort{api} 'si la interfa'ta administrativă).

Prin urmare, această lucrare arată că se poate concepe o solu'tie \acrshort{iot} la nivelul standardelor secolului 21, atât din punct de vedere al securită'tii cât 'și al proprietă'tii datelor stocate. Integrarea să u'soară poate doar să beneficieze adoptării mai rapide a acestui sistem.

Sinteza informa'tiilor conform literaturii de specialitate cât și analiza solu'tiilor existente din în ceea ce prive'ste ni'să încuietorilor inteligente a produs un exercițiu interesant în explorarea posibilității realizării unui sistem compatibil \acrshort{iot} care să ofere majoritatea functionalitatilor disponibile pe piață, păstrând în același timp spiritul hobist original.

S-a demonstrat că se pot folosi componente electronice simple pentru a realiza un optocuplor mai ieftin, dar care oferă accea'și func'tionalitate în cazul acestei aplica'tii. Prin ținerea costurilor finale, se oferă o propunere mai atractivă utilizatorilor doritori să încerce acest sistem.

Conceperea arhitecturii unui sistem complex care să satisfacă cerin'tele func'tionale prezentate de nișa dispozitivelor Smart Home din ziele noastre, culminata cu implementarea fizică a acestui concept 'si testarea în via'tă de zi cu zi demonstrează viabilitatea teoretică.

Prin testarea riguroasă a componentelor individuale ale sistemului, cât și testarea sa ca un întreg, se asigură un standard înalt de calitate pentru o aplicație cu o responsabilitate critică în viață utilizatorului final. Construind o fundație riguroasă pentru o platformă de tip Smart Lock care să ofere posibilitatea dezvoltărilor ulterioare, aducând valoare adăugată adepților timpurii.

\section{Dezvoltări ulterioare}

Dezvoltări ulterioare ale sistemului ar presupune construirea unui terminal \acrshort{pots} întreg ca \acrshort{hat} pentru RaspberryPi. Astfel vom putea atinge 'si cerin'ta de a oferi utilizatorului fluxuri audio pentru a comunica cu persoana de la celălalt capăt al interfonului. 