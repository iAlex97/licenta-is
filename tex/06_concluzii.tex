\section{Concluzii}

Având în vedere informa'tiile expuse în capitolele 1, 2, 3 'si cuno'stin'tele dobândite în realizarea arhitecturii unui sistem de tip încuietoare inteligentă, dar 'si rezultatele studiului de caz demonstrează că se pot crea solu'tii folosind tehnologii open-source. Prin alegeri sensibile din punct de vedere tehnic în ceea ce prive'ste securitatea sistemului, utilizatorul normal poate folosi aplica'tia pentru a î'si controla interfonul într-un mod sigur. De asemenea, includerea tehnologiilor open-source înseamnă că utilizatorii avansa'ti au multiple alegeri în ceea ce prive'ste implementarea solu'tiei (pot alege unde sunt stocate datele 'si cine are acces la ele, pot configura din firewall accesul la \acrshort{api} 'si la interfa'ta administrativă).

Prin urmare, această lucrare arată că se poate concepe o solu'tie \acrshort{iot} la nivelul standardelor secolului 21, atât din punct de vedere al securită'tii cât 'si al proprietă'tii datelor stocate. Integrarea sa u'soară poate doar să beneficieze adoptării mai rapide a acestui sistem.

\section{Contribu'tii}

\begin{itemize}
  \item Sinteza informa'tiilor în ceea ce prive'ste ni'sa încuietorilor inteligente.
  \item Analiza solu'tiilor existente din domeniu pentru în'telegerea pie'tei.
  \item Conceperea arhitecturii unui sistem similar care să satisfacă cerin'tele func'tionale.
  \item S-a demonstrat că se pot folosi componente electronice simple pentru a realiza un optocuplor mai ieftin, dar care oferă acceasi func'tionalitate în cazul acestei aplica'tii.
  \item Implementarea fizica a acestui concept 'si testarea în via'tă de zi cu zi.
\end{itemize}

\section{Dezvoltări ulterioare}

Dezvoltări ulterioare ale sistemului ar presupune construirea unui terminal \acrshort{pots} întreg ca \acrshort{hat} pentru RaspberryPi. Astfel vom putea atinge 'si cerin'ta de a oferi utilizatorului fluxuri audio pentru a comunica cu persoana de la celălalt capăt al interfonului. 