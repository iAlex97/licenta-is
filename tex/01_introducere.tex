\section {Obiectivele lucrării de licen'ta}

\subsection {Realizarea unui studiu de pia'tă pentru determinarea fezabilită'tii solu'tiei}

Pentru realizarea unui sistem \acrfull{iot} care să se plieze pentru necesită'tile secolului 21, este necesară o studiere mai aprofundată a caracteristicilor tehnice specificice. Astfel capitolul de introducere conturează ideile principale despre realizarea unui studiu de pia'tă pentru determinarea fezabilită'tii 'si pa'sii necesari pentru proiectarea unui sistem compatibil \acrshort{pots}.

În continuare se va realiza un scurt studiu de pia'tă pe ni'sa sistemelor \acrshort{iot} destinate uzului casnic. Un caz particular de astfel de dispozitive sunt cele care îndeplinesc func'tia de interfon sau oferă contrulul accesului într-o incinta de la distanta.

În momentul de fa'tă există pe pia'tă o multitudine de produse de tip încuietoare inteligentă sau sisteme tip interfon GSM, atât de la producători cunoscu'ti cât 'si de la branduri nou înfiin'tate. Această lucrare va analiza trei tipuri de solu'tii existente, cu implementări diferite, încercând să identifice func'tionalită'ti comune, avantaje 'si dezvantaje regăsite într-o plaja cât mai mare de dispozitive de pe pia'tă.

Situa'tia actuala din Romania prezintă o pia'tă cu o nevoie de îmbunătă'tire în ceea ce prive'ste sistemele actuale de interfon, deoarece majoritatea dintre ele au fost integrate în infrastructura blocurilor construite în trecut. Prin urmare exista un segment de pia'tă de utilizatori care ar dori să benefecieze de func'tiile intefonului inteligent, dar nu au această posibilitatea deoarece ar presupune schimbarea sistemului din întreaga clădire.


% \subsection {Dezvoltarea unui sistem compatibil POTS pentru interfa'tarea în re'teaua IoT}

Sistemul propus în această lucrare se poate conecta la re'teaua \acrfull{pots} printr-o simplă mufa RJ11 lucru ce ar trebui să u'sureze adoptarea unei îmbunătă'tiri la solu'tia actuala.


\section {Descrierea domeniului din care face parte proiectul de diplomă}

Unul dintre scopurile principale ale acestui proiect este eviden'tierea domeniului de automatizări \acrshort{iot} casnice adoptat din ce în ce mai des de consumatori domestici. Potrivit studiilor din domeniu, numărul de dispozitive \acrshort{iot} conectate la internet vă ajunge aproximativ la 75.44 miliarde în 2025 \cite{AlamTanweer2018}

\textit{Istoric}

Interesul în conectarea locuin'telor pentru a ob'tine func'tionalitate adi'tională datează încă din anii 60, majoritatea fiind concepte prototipate de entuzia'sti cu înclina'tii spre electronică.

Jim Sutherland, inginer la Westinghouse a creat primul sistem de automatizare a domiciliului în anul 1964, ECHO IV. Acesta era capabil să controleze temperatura, alte aparate casnice cât 'si să permită re'tinerea de mementouri sau liste de cumpărături. Cu introducerea re'telei \acrfull{arpanet} în 1969, un precursor al Internetului, universul dispozitivelor casnice conectate a cunoscut o perioadă rapidă de dezvoltare în anii următori \cite{ZeusIntegratedSystems}.

Trecerea de la o noutate scumpă la un sistem ce oferă func'tii cu adevărat practice a venit sub forma proiectului "X10 Home Automation". Se putea integra cu sistemul de climatizare existent al clădirii, controla electrocasnice mici, cât 'si corpuri de iluminat.

În anul 1984, Asocia'tia Na'tională a Constructorilor din Statele Unite a creat un grup de control numit "Smart House" pentru a accelera includerea tehnologiei în proiectele viitoare \cite{Aldrich2003Smart}.

Pentru consumatori, dezvoltările din următorii ani au adus u'si automate pentru garaje, termostate programabile 'si sisteme de securitate în cadrul monden, concomitent reducând preturile solu'tiilor oferite. În ciuda acestor semne, sociologii au concluzionat la vremea respectivă că nu există un interes real în conceptul "Smart House".


\textit{Stadiu actual}

În prezent solu'tiile de tip Smart Home oferă o multitudine de func'tionalită'ti, aduna 'si agregă informa'tii de la diferi'ti senzori plasa'ti în casă 'si agregă informa'tiile spre a fi afi'sat un rezumat utilizatorului. Printre informa'tiile monitorizate se pot 

Solu'tiile de tip "Smart Home" din prezent se integrează în general cu o re'tea precum Espressif, Apple HomeKit sau Google Home. Această permite controlul dispozitivelor conectate prin intermediul telefonului mobil \cite{RISTESKASTOJKOSKA20171454}.


\section {Prezentare pe scurt a capitolelor}

Capitolul 1 prezintă no'tiuni teoretice în scopul familiarizării cititorului în legătură cu solu'tiile actuale prezente pe pia'tă sistemelor \acrshort{iot}. Înainte de propunerea unei modalită'ti care să permită atingerea obiectivelor propuse, este necesar un studiu din trecut până în prezent asupra metodelor pe baza cărora să se poată contura o idee a ceea ce s-a putut obitine, până în acest moment pe pia'tă.

Copitolul 2 ilustrează perspectiva utilizatorului doritor de automatizarea tehnologică a lucrurilor ce îl înconjoară. Acest capitol se axează 'si pe o prezentarea scurtă a câtorva dispozitive ce îndeplinesc func'tia de încuietoare inteligentă, fiind comparate pentru a identifica punctele comune, dar 'si func'tiile unice ale sistemelor studiate.

Capitolul 3 începe prin prezentarea inova'tiilor tehnologice implementate 'si acceptate de către societate în prezent. Aceste inova'tii au fost men'tionate datorita aportului adus în schimbarea paradigmei procesării multiplelor surse de informa'tii provenite de la utilizatori pentru facilitarea activită'tilor zilnice. În urma analizei literaturii de specialitate se detaliază solu'tiile componete ale sistemului ce va fi implementat.

Pe baza no'tiunilor teoretice dobândite în capitolele anterioare, Capitolul 4 prezintă detalii tehnice cât 'si algoritmi utiliza'ti în solu'tia propusă. De asemenea se conturează pa'sii efectua'ti în conceperea metodei propuse pentru îndeplinirea scopului. A'sadar, în continuare lucrarea detaliază arhitectura sistemului, dezvoltarea unui \acrshort{hat} pentru Raspberry Pi, dar 'si componeta software alcătuită din server 'si clien'ti. Ultima etapă a procesului de dezvoltare a implicat testarea componentelor individuale cât 'si întregul ansamblu.

Capitolul 5 demonstrează 3 cazuri acoperite de func'tionalită'tile aplica'tiei din perspectiva utilizatorului.

Structura proiectului de diplomă se termină prin men'tionarea concluziilor 'si contri-bu'tiilor originale aduse în urma dobândirii cuno'stin'telor teoretice 'si practice. Faptul că tehnologia avansează cu pa'si rapizi 'si zilnic apar solu'tii noi, se vor contura de asemenea perspective viitoare de studiat.