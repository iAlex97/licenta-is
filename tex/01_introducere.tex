\section {Obiectivele lucrării de licența}

\subsection {Realizarea unui studiu de piata}

In continuare voi realiza un scurt studiu de piata pe nisa sistemelor \acrfull{iot} destinate uzului casnic. Un caz particular de astfel de dispozitive sunt cele care indeplinesc functia de interfon sau ofera contrulul accesului intr-o incinta de la distanta.

In momentul de fata exista pe piata o multitudine de produse de tip incuietoare inteligenta sau sisteme tip interfon GSM, atat de la producatori cunoscuti cat si de la branduri nou infiintate.

Dezavantajele solutiilor prezentate mai sus sunt faptul ca nu sunt proiectate sa fie integrate cu un sistem existent, intr-un bloc mai vechi. Prin urmare exista un segment de piata de utilizatori care ar dori sa benefecieze de functiile intefonului inteligent, dar nu pot deoarece asta ar presupune schimbarea sistemului din tot blocul.

\subsection {Dezvoltarea unui sistem compatibil POTS pentru interfatarea in reteaua IoT}

Pentru a putea oferi functiile inteligente unei audiente cat mai large, sistemul propus in aceasta lucrare se poate conecta la reteaua \acrfull{pots} printr-o simpla mufa RJ11.

\section {Descrierea domeniului din care face parte tema de licența}

Aceasta lucrare face parte dintr-un domeniu mai vechi, dar care a prins amploare recent, domeniul automatizarilor casnice si IoT. 

\subsection {Istoric}

Interesul in conectarea locuintelor pentru a obtine functionalitate aditionala dateaza inca din anii 60, majoritatea fiind concepte prototipate de entuziasti cu inclinatii spre electronica.

Jim Sutherland, inginer la Westinghouse a creat primul sistem de automatizare a domiciliului in anul 1964, ECHO IV. Acesta era capabil sa controleze temperatura, alte aparate casnice cat si sa permita retinerea de mementouri sau liste de cumparaturi. Cu introducerea retelei ARPAnet in 1969, un precursor al Internetului, universul dispozitivelor casnice conectate a cunoscut o perioada rapida de dezvoltare in anii urmatori \cite{ZeusIntegratedSystems}.

Trecerea de la novelty la un sistem ce ofera functii cu adevarat practice a venit sub forma proiectului "X10 Home Automation". Acesata se putea integra cu sistemul de climatizare existent al cladirii, controla electrocasnice mici, cat si corpuri de iluminat.

In anul 1984, Asociatia Nationala a Constructorilor din Statele Unite a creat un grup de control numit "Smart House" pentru a accelera includerea tehnologiei in proiectele viitoare \cite{Aldrich2003Smart}.

Pentru consumatori, dezvoltarile din urmatorii ani au adus usi automate pentru garaje, termostate programanbile sau sisteme de securitate in cadrul monden, concomitent reducand preturile solutiilor oferite. In ciuda acestor semne, sociologii au concluzionat la vremea respectiva ca nu exista un interes real in conceptul "Smart House".

\subsection {Curent}

Solutiile de tip "Smart Home" din prezent se integreaza in general cu o retea precum Espressif, Apple HomeKit sau Google Home. Aceasta permite controlul dispozitvelor conectate prin intermediul telefonului mobil.

\href{https://www.familyhandyman.com/article/the-history-of-smart-home-technology/}{Apple HomeKit/Google Home}

\href{https://techcrunch.com/2013/05/11/from-the-garage-to-200-employees-in-3-years-how-nest-thermostats-were-born/}{Nest TC}


\section {Prezentare pe scurt a capitolelor}

