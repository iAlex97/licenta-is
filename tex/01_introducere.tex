\section {Obiectivele lucrării de licența}

\subsection {Realizarea unui studiu de piata}

In continuare vom face un scurt studiu de piata pe nisa sistemelor \acrfull{iot} destinate uzului casnic. Un caz particular de astfel de dispozitive sunt cele care indeplinesc functia de interfon sau ofera contrulul accesului intr-o incinta de la distanta.

In momentul de fata exista pe piata o multitudine de produse de tip incuietoare inteligenta sau sisteme tip interfon GSM, atat de la producatori cunoscuti cat si de la branduri nou infiintate.

Dezavantajele solutiilor prezentate mai sus sunt faptul ca nu sunt proiectate sa fie integrate cu un sistem existent, intr-un bloc mai vechi. Prin urmare exista un segment de piata de utilizatori care ar dori sa benefecieze de functiile intefonului inteligent, dar nu pot deoarece asta ar presupune schimbarea sistemului din tot blocul.

\subsection {Dezvoltarea unui sistem compatibil POTS pentru interfatarea in reteaua IoT}

Pentru a putea oferi functiile inteligente unei audiente cat mai large, sistemul propus in aceasta lucrare se poate conecta la reteaua \acrfull{pots} printr-o simpla mufa RJ11.

\section {Descrierea domeniului din care face parte tema de licența}

\href{https://zeusintegrated.com/blog/item/a-brief-history-of-smart-home-automation}{History of smart home automation}

\href{https://www.familyhandyman.com/article/the-history-of-smart-home-technology/}{Apple HomeKit/Google Home}

\href{https://techcrunch.com/2013/05/11/from-the-garage-to-200-employees-in-3-years-how-nest-thermostats-were-born/}{Nest TC}

Studiu de caz: Nest si cum au crescut


Aceasta lucrare face parte dintr-un domeniu mai vechi, dar care a prins amploare recent, domeniul automatizarilor domestice (daca nu e industrial?) si IoT. 

\section {Prezentare pe scurt a capitolelor}

